\documentclass[a4paper]{article}

\usepackage[french]{babel}
\usepackage[utf8]{inputenc}
\usepackage{graphicx}
\usepackage{pdfpages}

\title{Small World : Documentation utilisateur \\ INSA de Rennes - 4INFO}

\author{Axel CARO, Maximilien RICHER}

\date{\today}

\begin{document}
\maketitle

\paragraph{}
Guide pour joueur
\\
\tableofcontents

\newpage

\section*{Introduction}
\paragraph{}
Le projet proposé consiste en la réalisation d'un jeu vidéo tour par tour, basé sur le jeu de plateau \textit{SmallWorld}. Au cours d'une partie, les joueurs s'affrontent pendant un certain nombre de tours, afin prendre le contrôle d'un territoire. Ce projet met l'accent sur les différentes phases de la réalisation d'un projet d'application orientée objet. La modélisation, l'utilisation de patrons de conceptions, l'organisation du code, l'implémentation, l'utilisation de librairies dynamiques, l'interfaçage entre plusieurs langages, la gestion d'IHM et les phases de tests sont des points clés des objectifs pédagogiques de ce projet.

\section{Présentation des règles du jeu}
Ceci est un rapide résumé des règles. Il précise les points non explicités dans le cahier des charges ainsi que certains choix de conception. Sa lecture est optionnelle si le lecteur est familier avec le sujet.

\subsection{But du jeu}
\paragraph{}
Le but de SmallWorld est d'accumuler des points de victoire.
À la fin de chaque tour de jeu, le joueur compte les points gagnés et les ajoute à son total. Au bout d'un nombre de tour défini au préalable, le jeu s'arrête et le joueur ayant le plus de points remporte la partie. Une partie peut également s'interrompre si un des joueurs élimine l'ensemble des unités de l'adversaire. Un joueur qui ne possède plus d'unités perd la partie, quel que soit son total de point.

\paragraph{}
\textbf{Déroulement d'une partie : }
\begin{itemize}
    \item Choix de la configuration du plateau de jeu
    \item Choix des races
    \item Choix du joueur qui commence la partie
    \item Création du plateau de jeu
    \item Début du premier tour de jeu\\(\dots)
    \item Fin du dernier tour de jeu
    \item Décompte final des points
\end{itemize}

\subsubsection{Plateau de jeu}
\paragraph{}
Le jeu se déroule sur un plateau quadrillé de cases carrées dont la dimension dépend du nombre de joueurs et de la durée de la partie.\label{map_gen} Il est composé de cases de différents types : plaine, mer, montagne et forêt. La carte est générée de manière aléatoire, mais doit contenir le même nombre de cases de chaque type.

\subsubsection{Unité}
\paragraph{}
Des unités sont placées sur le plateau, sur lequel elles peuvent être déplacées. Elles possèdent des points d'action, rechargés en début de tour, mais aussi une jauge de vie et des statistiques d'attaque et de défense.

\subsubsection{Races et décompte des points en fin de tour}
À la fin de chaque tour, le joueur compte les points gagnés et les ajoute à son total. Le montant des points gagnés est déterminé en fonction de la race choisie et du type de terrain sur lequel ses unités sont placées.

\paragraph{}
\textbf{Configurations conseillées :}
\begin{itemize}
    \item 2 joueurs, plateau de 6 par 6, 5 tours de jeu : 4 unités par joueur
    \item 2 joueurs, plateau de 10 par 10, 20 tours de jeu : 6 unités par joueur
    \item 2 joueurs, plateau de 14 par 14, 30 tours de jeu : 8 unités par joueur
\end{itemize}

\subsection{Déroulement d'un tour de jeu}
\paragraph{}
Chaque \textit{tour de jeu} est constitué de plusieurs \textit{tours de joueurs}. Ces tours sont séquentiels. Lors du premier tour de jeu, le joueur jouant le premier tour est tiré au hasard, et cet ordre est conservé durant le reste de la partie.

\paragraph{}
\textbf{Déroulement du tour : }
\begin{itemize}
    \item Début du tour
    \item Déplacement des unités et combats
    \item Décompte des points
    \item Fin du tour
\end{itemize}

\subsection{Déplacement et combat}
\paragraph{}
Un joueur peut, durant son tour, déplacer une ou plusieurs unités lui appartenant. Une unité ne peut être déplacée que s'il lui reste assez de points d'action pour effectuer le déplacement demandé. Le nombre de points d'action nécessaires dépend de la race du joueur et du type des cases traversées. Les déplacements ne peuvent s'effectuer que vers une case adjacente, et non en diagonale. Un déplacement vers une case occupée par une ou plusieurs unités alliées est autorisé. Un déplacement vers une case occupée par une unité adverse engendre un combat.

\paragraph{}
Il est possible d'attaquer une unité plusieurs fois de suite, à condition de disposer des points d'action nécessaires.

\subsubsection{Combat au corps à corps}
Le combat au corps à corps est provoqué par un déplacement. Dans le cas de la mort de l'unité adverse, l'attaquant est déplacé sur sa case si aucune autre unité adverse ne s'y trouve encore. Dans le cas contraire, il conserve sa position.
Lors de son attaque, il engendre une riposte de l'adversaire. Cette riposte n'influe pas sur les points d'actions dont disposera l'unité du défenseur au tour suivant.
\subsubsection{Combat à distance}
Un combat à distance n'implique pas de déplacement. L'attaquant doit se trouver sur la même ligne ou la même colonne que sa cible, à deux cases de distance. Sur la case vide situé entre l'attaquant et sa cible peut se trouver une unité allié ou adverse sans que cela n'ai d'incidence sur le déroulement de l'attaque.
Ce combat se déroule sans contre-attaque de l'adversaire si celui-ci est de type corps à corps. S'il s'agit d'une unité de type distance, elle ripostera. Cependant, aucun déplacement n'est engendré en cas de victoire.
\section{Modélisation UML}
\paragraph{}
La phase de modélisation nous a conduit à réaliser plusieurs diagrammes UML, afin de représenter différents aspects de notre application. Nous présentons ci-après ces diagrammes, en explicitant leur contenu.

\end{document}