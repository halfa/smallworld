\documentclass[a4paper]{article}

\usepackage[french]{babel}
\usepackage[utf8]{inputenc}
\usepackage{graphicx}
\usepackage{pdfpages}

\title{Small World : Rapport de Modélisation \\ INSA de Rennes - 4INFO}

\author{Axel CARO, Maximilien RICHER}

\date{\today}

\begin{document}
\maketitle

\paragraph{}
Ce document présente les résultats de la phase de modélisation du projet de quatrième année proposé en enseignement de \textit{Programmation et Modélisation orientées objet}.

\section*{Introduction}
\paragraph{}
Le projet proposé consiste en la réalisation d'un jeu vidéo tour par tour, basé sur le jeu de plateau \textit{SmallWorld}. Au cours d'une partie, les joueurs s'affrontent pendant un certain nombre de tours, afin prendre le contrôle d'un territoire. Ce projet met l'accent sur les différentes phases de la réalisation d'un projet d'application orientée objet. La modélisation, l'utilisation de patrons de conceptions, l'organisation du code, l'implémentation, l'utilisation de librairies dynamiques, l'interfaçage entre plusieurs langages, la gestion d'IHM et les phases de tests sont des points clés des objectifs pédagogiques de ce projet.

\section{Présentation des règles du jeu}

\subsection{But du jeu}
Le but de SmallWorld est d'accumuler des points. A la fin de chaque tour de jeu, le joueur compte les points gagnés et les ajoutes à son total. Au bout d'un nombre de tour défini au préalable, le jeu s'arrête et le joueur ayant le plus de point remporte la partie.
\subsubsection{Plateau de jeu}
Le jeu se déroule sur une plateau dont la dimension dépend du nombre de joueurs et de la durée de la partie.

\subsection{Déroulement d'un tour de jeu}

\begin{itemize}
    \item 
\end{itemize}

\subsection{Déplacement et combat}


\section{Modélisation UML}
\paragraph{}
La phase de modélisation nous a conduit à réaliser plusieurs diagrammes UML, afin de représenter différents aspects de notre application. Nous présentons ci-après ces diagrammes, en explicitant leur contenu.

\subsection{Diagramme de paquetage}
\paragraph{}
Le diagramme suivant illustre la structuration à gros grain du code de notre application. On y retrouve les classes et interfaces que nous détaillerons dans la section \ref{DDC}. Ce diagramme illustre les interactions entre les regroupements de classes. On ne parle pas ici de \textit{package} proprement dit, car le choix du \textbf{C\#} comme langage de programmation ne permet pas l'utilisation de paquetages. Il s'agit d'un découpage ayant pour but de présenter la structure des différentes parties de l'application, sans détailler les interactions internes à chaque regroupement.

\paragraph{}
On distingue 7 regroupements, en plus de l'application principale (\textit{SmallWorld}) :
\begin{enumerate}
\item \textit{Tile} : gestion des cases du terrain de jeu.
\item \textit{Map} : gestion du terrain de jeu.
\item \textit{Race} : gestion des différentes races jouables.
\item \textit{Unit} : gestion des unités.
\item \textit{Player} : gestion des joueurs.
\item \textit{Game} : gestion de la partie.
\item \textit{SaveLoad} : gestion de la sauvegarde et du chargement de parties.
\end{enumerate}

\paragraph{}
Les interactions entre ces différents regroupements sont symbolisées par les flèches présentes sur le diagramme. Par exemple, le groupement \textit{Map} utilise des membres du groupement \textit{Tile}. Les interactions ainsi mises en valeur présentent bien la mise en cascade des regroupements, selon les données modélisées.

\includepdf[landscape=true]{images/packageDiagram.pdf}

\subsection{Diagramme de classes}
\label{DDC}
\paragraph{}
Le diagramme suivant représente l'organisation en classes de notre application.

\includepdf{images/classDiagram.pdf}

\subsection{Diagrammes d'interaction}

\subsubsection{Diagramme de séquence : initialisation d'une partie}

\subsubsection{Diagramme de séquence : déroulement d'un tour}

\subsection{Diagramme d'états-transitions : gestion d'une unité}

\end{document}
